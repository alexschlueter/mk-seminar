\documentclass[a4paper,11pt]{scrartcl}
\usepackage{lmodern}
\usepackage[utf8]{inputenc}
\usepackage[ngerman]{babel}
\usepackage[T1]{fontenc}
\usepackage{amsmath,amsthm,amssymb}
\usepackage{mathtools}
\usepackage{braket}

\theoremstyle{plain}
\newtheorem{thm}{Satz}[section]
\newtheorem{lem}[thm]{Lemma}
\newtheorem{prop}[thm]{Proposition}
\newtheorem{cor}[thm]{Korollar}

\theoremstyle{definition}
\newtheorem{defn}[thm]{Definition}
\newtheorem{conj}[thm]{Vermutung}
\newtheorem{exmp}[thm]{Beispiel}

% \theoremstyle{remark}
\newtheorem{rem}[thm]{Bemerkung}
\newtheorem*{note}{Merke}

\newcommand{\stcomp}[1]{{#1}^{\mathsf{c}}} % Mengenkomplement
\DeclarePairedDelimiter{\sprod}{\langle}{\rangle}	% spitze Klammern
\DeclarePairedDelimiter{\abs}{\lvert}{\rvert}		% Betrag

\begin{document}

\title{Mischungsverhalten und Eigenfunktionen}
% \title{Abschätzung der Spektrallücke über Leitfähigkeit}
\author{Alexander Schlüter}
% \thanks{Betreuung: Prof. Dr. Dereich, Johannes Blank}
\date{18. Januar 2016}
\maketitle

\section{Einleitung}
Wir haben schon die Leitfähigkeit einer Markovkette als eine Möglichkeit zur
Abschätzung ihrer Mischzeit kennengelernt. In diesem Vortrag soll die
Spektrallücke einer Markovkette anhand der Eigenwerte ihrer Übergangsmatrix
definiert werden. Das Hauptresultat ist eine Abschätzung der Spektrallücke
mithilfe der Leitfähigkeit.

\subsection{Erinnerungen}
Im Folgenden sei $\Omega$ ein endlicher Zustandsraum, $P$ eine irreduzible und
aperiodische Übergangsmatrix auf $\Omega$ mit stationärer Verteilung $\pi$.
\begin{defn}
 Schreibe für $x, y\in\Omega$ 
 \begin{equation*}
  Q(x,y)\coloneqq\pi(x)P(x,y).
 \end{equation*}
 Für $A, B\subset\Omega$ sei das \textbf{Randmaß} $Q$ definiert durch
 \begin{equation*}
   Q(A,B)\coloneqq\sum_{x\in A, y\in B}Q(x,y). 
 \end{equation*}
 Die \textbf{Leitfähigkeit} einer Menge $S$ ist
 \begin{equation*}
  \phi(S)\coloneqq\frac{Q(S,\stcomp{S})}{\pi(S)} 
 \end{equation*}
 und die Leitfähigkeit der ganzen Kette
 \begin{equation*}
  \phi_\star\coloneqq\min\Set{\phi(S)\mid S\subset\Omega,\,\pi(S)\leq\frac{1}{2}}.
 \end{equation*}
\end{defn}
Wir haben gesehen, dass die Leitfähigkeit die Mischzeit der Markovkette
beeinflusst. Markovketten mit einer geringen Leitfähigkeit mischen langsamer,
d.h. sie konvergieren langsamer gegen die stationäre Verteilung. Man kann sich
auch vorstellen, dass die Leitfähigkeit die ``engste Stelle'' der Kette misst,
daher der Name ``Flaschenhals-Quotient''.

\section{Eigenwerte und die Spektrallücke}
Die Verteilung einer Markovkette wird im Wesentlichen durch ihre Übergangsmatrix
charakterisiert, deshalb ist es sinnvoll, sich diese mithilfe der Linearen
Algebra genauer anzuschauen.

Sei im Folgenden $P$ \textbf{reversibel} bezüglich $\pi$.
\begin{defn}
  Seien $f, g\in\mathbb{R}^\Omega$. Definiere
  \begin{equation*}
    \sprod{f,g}_\pi\coloneqq\sum_{x\in\Omega}f(x)g(x)\pi(x)
  \end{equation*}
\end{defn}

\begin{rem}
 Da für alle $x\in\Omega\quad\pi(x)>0$ gilt, wird hierdurch ein Skalarprodukt auf
 $\mathbb{R}^\Omega$ definiert. Der Hilbertraum $(\mathbb{R}^\Omega,
 \sprod{\cdot,\cdot})$ hat eine Orthonormalbasis aus reellen Eigenfunktionen
 $f_1,\dots, f_{\abs{\Omega}}$ wobei $f_i$ Eigenfunktion zum Eigenwert
 $\lambda_i$ ist ($1\leq i\leq\abs{\Omega}$). 
\end{rem}

\end{document}