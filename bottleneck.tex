\documentclass[ngerman,a4paper,11pt]{scrartcl}
\usepackage{lmodern}
\usepackage[utf8]{inputenc}
\usepackage{babel}
\usepackage{csquotes}
\usepackage{bbm}
\usepackage[T1]{fontenc}
\usepackage{braket}
\usepackage{enumitem}
\usepackage[shortcuts]{extdash}
\usepackage{cancel}
\usepackage[backend=biber,style=alphabetic]{biblatex}
\usepackage{nameref}
\usepackage{hyperref}
\usepackage{amsmath,amsthm,amssymb}
\usepackage{mathtools}
\usepackage{thmtools}
\usepackage[capitalize]{cleveref}

\bibliography{literatur.bib}

\newlist{thmlist}{enumerate}{1}
\setlist[thmlist]{label=(\roman{thmlisti}), ref=\thethm (\roman{thmlisti}),noitemsep}
\newlist{remlist}{enumerate}{1}
\setlist[remlist]{label=(\arabic*), ref=\thethm (\arabic*),noitemsep}

\newtheoremstyle{break}{}{}{}{}{\bfseries}{}{\newline}{}
\declaretheoremstyle[headpunct=.:]{claim}

\declaretheorem[
name=Beispiel,
%refname={theorem,theorems},        %Lower Case Versions of Theorem Type
% Refname={Satz,Sätze},
style=break,
numberwithin=section]{exmpbk}

\declaretheorem[
name=Satz,
%refname={theorem,theorems},        %Lower Case Versions of Theorem Type
% Refname={Satz,Sätze},
style=break,
numberwithin=section]{thmbk}

\declaretheorem[
name=Satz,
%refname={theorem,theorems},        %Lower Case Versions of Theorem Type
% Refname={Satz,Sätze},
style=definition,
numberwithin=section]{thm}

\declaretheorem[
name=Lemma,
%refname={lemma,lemmas},
% Refname={Lemma,Lemmas},
style=definition,
sibling=thm]{lem}

\declaretheorem[
name=Definition,
%refname={lemma,lemmas},
% Refname={Definition,Definitionen},
style=definition,
sibling=thm]{defn}

\declaretheorem[
name=Bemerkung,
%refname={lemma,lemmas},
% Refname={Bemerkung,Bemerkungen},
style=definition,
sibling=thm]{rem}

\declaretheorem[
name=Beispiel,
%refname={lemma,lemmas},
% Refname={Bemerkung,Bemerkungen},
style=definition,
sibling=thm]{exmp}

\declaretheorem[
name=Beh,
numbered=no,
%refname={lemma,lemmas},
% Refname={Bemerkung,Bemerkungen},
style=claim]{claim}

\declaretheorem[
name=Beweis,
numbered=no,
%refname={lemma,lemmas},
% Refname={Bemerkung,Bemerkungen},
qed=\ensuremath{\Diamond},
style=remark]{dproof}

% \Crefname{thm}{Satz}{Sätze}
% \Crefname{lem}{Lemma}{Lemmas}

\addtotheorempostheadhook[thm]{\crefalias{thmlisti}{thm}}
\addtotheorempostheadhook[lem]{\crefalias{thmlisti}{lem}}
\addtotheorempostheadhook[rem]{\crefalias{remlisti}{rem}}

% \theoremstyle{definition}
% \newtheorem{thm}{Satz}[section]
% \newtheorem{lem}[thm]{Lemma}
% \newtheorem{prop}[thm]{Proposition}
% \newtheorem{cor}[thm]{Korollar}

% \newtheorem{defn}[thm]{Definition}
% \newtheorem{conj}[thm]{Vermutung}
% \newtheorem{exmp}[thm]{Beispiel}

% % \theoremstyle{remark}
% \newtheorem{rem}[thm]{Bemerkung}
% \newtheorem*{note}{Merke}
\DeclarePairedDelimiterX{\norm}[1]{\lVert}{\rVert}{#1}
\newcommand{\fracnorm}[3]{%
    \ensuremath{\frac{\! #1\!}{\! {}_{\phantom{#3}}\norm{#2}_{#3}\!}}%
}
\newcommand\coloniff{\vcentcolon\!\iff}

\makeatletter
\newcommand*{\transpose}{%
  {\mathpalette\@transpose{}}%
}
\newcommand*{\@transpose}[2]{%
  % #1: math style
  % #2: unused
  \raisebox{\depth}{$\m@th#1\intercal$}%
}
\makeatother

\newcommand{\stcomp}[1]{{#1}^{\mathsf{c}}} % Mengenkomplement
\newcommand{\diri}{\mathcal{E}}
\newcommand{\CC}{\mathbb{C}}
\newcommand{\FF}{\mathbb{F}}
\newcommand{\HH}{\mathbb{H}}
\newcommand{\KK}{\mathbb{K}}
\newcommand{\NN}{\mathbb{N}}
\newcommand{\OO}{\mathbb{O}}
\newcommand{\QQ}{\mathbb{Q}}
\newcommand{\RR}{\mathbb{R}}
\newcommand{\ZZ}{\mathbb{Z}}
\DeclarePairedDelimiter{\sprod}{\langle}{\rangle}	% spitze Klammern
\DeclarePairedDelimiter{\abs}{\lvert}{\rvert}		% Betrag
\DeclareMathOperator{\ggT}{ggT}

\newcommand{\smallJ}{{\mathchoice{}{}{\scriptscriptstyle}{}J}}
\newcommand{\smallS}{{\mathchoice{}{}{\scriptscriptstyle}{}S}}
\newcommand{\smallI}{{\mathchoice{}{}{\scriptscriptstyle}{}I}}

\newcommand\phantomarrow[2]{%
  \setbox0=\hbox{$\displaystyle #1\to$}%
  \hbox to \wd0{%
    $#2\mapstochar
     \cleaders\hbox{$\mkern-1mu\relbar\mkern-3mu$}\hfill
     \mkern-7mu\rightarrow$}%
  \,}

\begin{document}

% \title{Mischungsverhalten und Eigenfunktionen}
\title{Abschätzung der Spektrallücke über die Leitfähigkeit}
\author{Alexander Schlüter}
% \thanks{Dozent: Prof.~Dr.~Dereich, Betreuung: Johannes Blank}
\date{18. Januar 2016}
% Bachelorseminar Markovketten, WS 15/16
% {\let\newpage\relax\maketitle}
\maketitle
\begin{abstract}
Wir haben schon die Leitfähigkeit einer Markovkette als eine Möglichkeit zur
Abschätzung ihrer Mischzeit kennengelernt. In diesem Vortrag soll die
Spektrallücke einer Markovkette anhand der Eigenwerte ihrer Übergangsmatrix
definiert werden. Das Hauptresultat ist eine Abschätzung der Spektrallücke
mithilfe der Leitfähigkeit.
\end{abstract}
% \newpage
\tableofcontents
% \newpage

\section{Erinnerungen}

% \subsection{Erinnerungen}
Im Folgenden sei $\Omega$ ein endlicher Zustandsraum mit mehr als einem Element, $P$ eine irreduzible Übergangsmatrix auf $\Omega$ mit stationärer Verteilung $\pi$.
\begin{defn}
 Schreibe für $x, y\in\Omega$ 
 \begin{equation*}
  Q(x,y)\coloneqq\pi(x)P(x,y).
 \end{equation*}
 Für $A, B\subset\Omega$ sei das \textbf{Kantenmaß} $Q$ definiert durch
 \begin{equation*}
   Q(A,B)\coloneqq\sum_{x\in A, y\in B}Q(x,y). 
 \end{equation*}
 Die \textbf{Leitfähigkeit} einer nichtleeren Menge $S\subset\Omega$ ist
 \begin{equation*}
  \phi(S)\coloneqq\frac{Q(S,\stcomp{S})}{\pi(S)} 
 \end{equation*}
 und die Leitfähigkeit der ganzen Kette
 \begin{equation*}
  \phi_\star\coloneqq\min\Set{\phi(S)\mid S\subset\Omega,\,0<\pi(S)\leq\frac{1}{2}}.
 \end{equation*}
\end{defn}
Wir haben gesehen, dass die Leitfähigkeit die Mischzeit der Markovkette
beeinflusst. Markovketten mit einer geringen Leitfähigkeit mischen langsamer,
d.h. sie konvergieren langsamer gegen die stationäre Verteilung. Man kann sich
auch vorstellen, dass die Leitfähigkeit die \enquote{engste Stelle} der Kette misst,
daher der Name \enquote{Flaschenhals-Quotient}.

\section{Eigenwerte und die Spektrallücke}
Die Verteilung einer Markovkette wird im Wesentlichen durch ihre Übergangsmatrix
charakterisiert, deshalb ist es sinnvoll, diese mit Mitteln der Linearen
Algebra genauer zu betrachten.

% Sei im Folgenden $P$ \textbf{reversibel} bezüglich $\pi$.
\begin{defn}
  Seien $f, g\in\RR^\Omega$. Definiere
  \begin{align*}
    \sprod{f,g}_\pi&\coloneqq\sum_{x\in\Omega}f(x)g(x)\pi(x) \\
    \norm{f}_\pi&\coloneqq\sqrt{\sprod{f,f}} \, .
  \end{align*}
 Schreibe außerdem
 \begin{equation*}
  f\perp_\pi g\coloniff\sprod{f,g}_\pi=0\, .
 \end{equation*}
 $f$ und $g$ heißen dann \textbf{$\pi$\=/orthogonal}.
\end{defn}

\begin{rem}
 Da für alle $x\in\Omega\quad\pi(x)>0$ gilt, wird hierdurch ein Skalarprodukt auf
 $\RR^\Omega$ definiert und $\ell^2(\pi)\coloneqq(\RR^\Omega,
 \sprod{\cdot,\cdot}_\pi)$ ist ein Hilbertraum.
\end{rem}

\begin{lem}
  Sei $M$ eine Übergangsmatrix.
  \begin{thmlist}
  \item Ist $\lambda$ ein Eigenwert von $M$, dann gilt $\abs{\lambda}\leq 1$.
  \item Ist $M$ irreduzibel, dann ist der Eigenraum zum Eigenwert $1$
    eindimensional und er wird aufgespannt durch $(1,1,\dotsc,1)^\transpose$.
  \item Ist $M$ irreduzibel und aperiodisch, so ist $-1$ kein Eigenwert.\label{lem:apew}
  \end{thmlist}
\end{lem}
\begin{proof}
  \begin{thmlist}
  \item Sei $f$ Eigenfunktion zu $\lambda$ und $x\in\Omega$ mit
    $\abs{f(x)}=\max_{y\in\Omega}\abs{f(y)}$. Dann gilt
    \begin{equation*}
     \abs{\lambda f(x)}=\abs{\sum_{y\in\Omega}M(x,y)f(y)}\leq\abs{f(x)}\sum_{y\in\Omega}M(x,y)=\abs{f(x)} 
    \end{equation*}
    und da f nicht konstant $0$ ist, folgt $\abs{\lambda}\leq 1$.
  \item Da die Zeilensumme $1$ ist, ist $(1,1,\dotsc,1)^\transpose$ offensichtlich
    Eigenfunktion zum Eigenwert $1$.
    Sei $f$ Eigenfunktion zu $1$ und $x\in\Omega$ mit
    $\abs{f(x)}=\max_{y\in\Omega}\abs{f(y)}$. Dann 
    \begin{equation*}
     \sum_{y\in\Omega}M(x,y)\frac{f(y)}{f(x)}=1 
    \end{equation*}
    mit $\frac{f(y)}{f(x)}\leq 1$. Es muss also für alle $y\in\Omega$ mit
    $M(x,y)>0$\quad$\frac{f(y)}{f(x)}=1$ gelten. 

Ist $y\in\Omega$ beliebig, so existiert $t\in\NN$ mit $M^t(x,y)>0$, denn
$M$ ist irreduzibel. $f$ ist auch Eigenfunktion von $M^t$ zum Eigenwert $1$ und mit dem gleichen Argument
wie oben für $M^t$ folgt $f(y)=f(x)$.
  \item Zeige die Kontraposition: Sei $f$ Eigenfunktion zum
    Eigenwert $-1$. Sei $x\in\Omega$ mit $\abs{f(x)}=\max_{y\in\Omega}\abs{f(y)}$.
    \begin{equation*}
     \abs{-f(x)}=\abs{\sum_{y\in\Omega}M(x,y)f(y)}\leq\sum_{y\in\Omega}M(x,y)\abs{f(y)}\leq\abs{f(x)}
    \end{equation*}
    und da $\abs{f(y)}\leq\abs{f(x)}$ folgt für alle $y\in\Omega$ mit
    $M(x,y)>0$, dass $\abs{f(y)}=\abs{f(x)}$. 
    % Da $M$ irreduzibel ist, folgt $\abs{f(y)}=r$ für alle $y\in\Omega$.

    Ohne Beträge sieht man, dass
    \begin{equation*}
     -f(x)=\sum_{y\in\Omega}M(x,y)f(y)
    \end{equation*}
   was nur möglich ist, wenn für $y\in\Omega$ mit $M(x,y)>0$ gilt: $f(y)=-f(x)$. 
   
   Sei $t\in\mathcal{T}(x)\coloneqq\Set{s\in\NN\mid M^s(x,x)>0}$. Es existiert
   ein Pfad
   \begin{equation*}
     x=v_0\rightarrow v_1\rightarrow v_2\rightarrow\dotso\rightarrow v_{t-1}\rightarrow v_t=x
   \end{equation*}
   mit $v_i\in\Omega$, $M(v_i, v_{i+1})>0$ ($0\leq i\leq t-1$). Nach obiger
   Betrachtung ist 
   \begin{equation*}
    f(x)=-f(v_{t-1})=\dotsb=(-1)^tf(x) 
   \end{equation*}
   Also muss $t$ gerade sein und $\ggT(\mathcal{T}(x))\geq 2$, d.h. $M$ ist
   periodisch.
  \end{thmlist}
\end{proof}

\begin{lem}
 \label{lem:ortho}
 Sei $P$ reversibel bezüglich $\pi$. Dann hat $\ell^2(\pi)$  eine
 Orthonormalbasis $f_1,\dotsc ,f_{\abs{\Omega}}$, wobei $f_i$ Eigenfunktion von
 $P$ zum Eigenwert $\lambda_i$ ($1\leq i\leq\abs{\Omega}$) ist. Die
 Eigenfunktion zum Eigenwert $1$ kann als die konstante Einsfunktion $f_1=\mathbf{1}$ gewählt werden.
\end{lem}
\begin{proof}
 Aus Reversibilität folgt, dass die Matrix
 \begin{equation*}
  A(x,y)\coloneqq\pi(x)^{1/2}\pi(y)^{-1/2}P(x,y) 
 \end{equation*}
 symmetrisch ist. Die Behauptung folgt mittels des Spektralsatzes für
 symmetrische Matrizen. Für den vollen Beweis siehe \cite[Lemma~12.2(i)]{lpw}.
\end{proof}

\begin{rem}
 \label{rem:eigortho}
 \begin{remlist}
 \item Da $A$ in dem Beweis von \cref{lem:ortho} symmetrisch ist, sind Eigenfunktionen
  zu unterschiedlichen Eigenwerten orthogonal. Dies überträgt sich auf die
  Eigenfunktionen von $P$: Eigenfunktionen von $P$ zu unterschiedlichen
  Eigenwerten sind \textbf{$\pi$\=/orthogonal}.
 \item Die Eigenfunktionen in der Basis sind im Allgemeinen nicht eindeutig bestimmt, die
   Vielfachheit der auftretenden Eigenwerte aber schon: Sie entspricht der
   Dimension des zugehörigen Eigenraumes von $P$. \label{rem:eignum}
 \end{remlist}
\end{rem}

\begin{defn}
 Für eine reversible Übergangsmatrix $P$ nummerieren wir ihre Eigenwerte in
 absteigender Ordnung:
 \begin{equation*}
  1=\lambda_1>\lambda_2\geq\dotsb\geq\lambda_{\abs{\Omega}}\geq -1.
 \end{equation*}
 Die \textbf{Spektrallücke} ist definiert durch $\gamma\coloneqq 1-\lambda_2$.
 Schreibe
 \begin{equation*}
  \lambda_\star\coloneqq\max\Set{\abs{\lambda}\mid\text{$\lambda$ ist Eigenwert von $P$, $\lambda\neq 1$}}. 
 \end{equation*}
  Die Differenz $\gamma_\star\coloneqq 1-\lambda_\star$ heißt \textbf{absolute Spektrallücke}.
\end{defn}

\begin{rem}
 Nach \cref{lem:apew} ist für eine irreduzible aperiodische Übergangsmatrix $\gamma_\star>0$. 
\end{rem}

\section{Die Dirichletform}
Bis jetzt ist noch nicht klar, was die Spektrallücke mit der Leitfähigkeit zu
tun hat. Die Verbindung wird über die Dirichletform hergestellt.

Ab hier sei die irreduzible Übergangsmatrix $P$ zusätzlich \textbf{reversibel} bezüglich der
stationären Verteilung $\pi$.

\begin{defn}
  % Sei $P$ eine reversible Übergangsmatrix mit stationärer Verteilung $\pi$.
  Die Abbildung
  \begin{align*}
   \diri \colon &\RR^\Omega\times\RR^\Omega \to\RR \\
   &\phantomarrow{\RR^\Omega\times\RR^\Omega}{(f,g)}\sprod{(I-P)f,g}_\pi
   % &(f,g)\xmapsto{\phantom{\RR^\Omega\times\RR^\Omega}}\sprod{(I-P)f,g}_\pi
  \end{align*}
  heißt \textbf{Dirichletform} zu $(P,\pi)$.
\end{defn}

\begin{lem}
 \label{lem:diri}
 Definiere für $f\in\RR^\Omega$
 \begin{equation}
  \label{eq:diri}
  \diri(f)\coloneqq\frac{1}{2}\sum_{x,y\in\Omega}(f(x)-f(y))^2\pi(x)P(x,y).
 \end{equation}
 Dann gilt $\diri(f)=\diri(f,f)$.
\end{lem}
\begin{proof}
  Ausmultiplizieren des Quadrats in \cref{eq:diri} gibt
  \begin{equation*}
   \diri(f)=\underbrace{\frac{1}{2}\sum_{x,y\in\Omega}f(x)^2\pi(x)P(x,y)}_{\text{(a)}}-
   \underbrace{\sum_{x,y\in\Omega}f(x)f(y)\pi(x)P(x,y)}_{\text{(b)}}
   +\underbrace{\frac{1}{2}\sum_{x,y\in\Omega}f(y)^2\pi(x)P(x,y)}_{\text{(c)}}
  \end{equation*}
  Wegen Reversibilität ist $\pi(x)P(x,y)=\pi(y)P(y,x)$ und die Terme (a) und (c)
  lassen sich zusammenfassen:
  \begin{align*}
   (a)+(c)&=\sum_{x,y\in\Omega}f(x)^2\pi(x)P(x,y) \\
   &=\sum_{x\in\Omega}f(x)^2\pi(x)\sum_{y\in\Omega}P(x,y) \\
   &=\sum_{x\in\Omega}f(x)^2\pi(x).
  \end{align*}
  Zusammen mit (b) also
  \begin{align*}
   \diri(f)&=\sum_{x\in\Omega}f(x)^2\pi(x)-\sum_{x\in\Omega}\left(\sum_{y\in\Omega}P(x,y)f(y) \right)f(x)\pi(x) \\
   &=\sprod{f,f}_\pi-\sprod{Pf,f}_\pi \\
   &=\sprod{(I-P)f,f}_\pi \\
   &=\diri(f,f)
  \end{align*}
\end{proof}

\begin{rem}
 Es sei $\mathbf{1}$ die
 konstante Einsfunktion auf $\Omega$. Dann
 ist
 \begin{equation*}
  E_\pi(g)=\sum_{x\in\Omega}g(x)\pi(x)=\sprod{g,\mathbf{1}}_\pi\, ,
 \end{equation*}
 also $E_\pi(g)=0$ genau dann wenn $g\perp_\pi\mathbf{1}$.
\end{rem}

\begin{lem}
 \label{lem:spekdiri}
 Für die Spektrallücke $\gamma=1-\lambda_2$  gilt
 \begin{align}
  \gamma&=\min\Set{\diri(g)\mid g\in\RR^{\Omega} , \, g\perp_\pi\mathbf{1} , \, \norm{g}_\pi=1} \label{eq:min} \\
        &=\min\Set{\frac{\diri(g)}{\norm{g}_\pi^2}\mid g\in\RR^\Omega,\, g\perp_\pi\mathbf{1},\, g\neq 0} \label{eq:min2}
 \end{align}
\end{lem}

% Unnötig, weil Beweis Minimum zeigt
% \begin{rem}
%  Die Menge $\Set{g\in\RR^\Omega\mid\norm{g}_\pi=1}$ ist kompakt und die Menge
%  $E_\pi^{-1}(\{0\})$ ist abgeschlossen, der Schnitt ist also kompakt. Da die Abbildung $\diri(\cdot)$ stetig
%  ist, hat die Menge in \cref{eq:min} tatsächlich ein Minimum. Im Beweis wird
%  gezeigt, dass die Menge mit der in \cref{eq:min2} übereinstimmt. Darum ist auch
%  hier das Minimum gerechtfertigt.
% \end{rem}

% \begin{rem}
%  Die Dirichletform ist eine Art Energiefunktional (oder Potential).
% \end{rem}

\begin{proof} % [Beweis von \cref{lem:spekdiri}]
 Sei $n\coloneqq\abs{\Omega}$. Laut \cref{lem:ortho} gibt es in $\ell^2(\pi)$ eine
 Orthonormalbasis $f_1,\dotsc,f_n$ von Eigenfunktionen von $P$, wobei $f_1=\mathbf{1}$. Ist
 $g\in\RR^\Omega$, so kann $g$ in der Orthonormalbasis entwickelt werden zu
 \begin{equation*}
  g=\sum_{j=1}^n\sprod{g,f_j}_\pi f_j\, . 
 \end{equation*}
 Setze $a_j\coloneqq\sprod{g,f_j}_\pi$.
 Ist $g\perp_\pi\mathbf{1}$, d.h. $\sprod{g,f_1}_\pi=0$, so fällt der erste
 Summand weg. Gilt auch $\norm{g}_\pi=1$, dann ist
 \begin{equation*}
  1=\norm{g}_\pi^2=\sprod{g,g}_\pi=\sum_{j=2}^na_j^2\, .
 \end{equation*}
 Insgesamt
 \begin{align*}
  \diri(f)&=\sprod{(I-P)g,g}_\pi=\sprod{\, (I-P)\sum_{j=2}^n a_j f_j\, ,\, g\,}_\pi \\
          &=\sprod{\, \sum_{j=2}^n a_j (1-\lambda_j)f_j\, ,\, g\, }_\pi=\sum_{j=2}^na_j^2 (1-\lambda_j) \\
          &\geq(1-\lambda_2)\sum_{j=2}^n a_j^2=(1-\lambda_2)
 \end{align*}
 mit Gleichheit im vorletzten Schritt für $g=f_2$. Damit ist \cref{eq:min} gezeigt.
 
 Die Ungleichung \enquote{$\geq$} von \eqref{eq:min} zu \eqref{eq:min2} ist klar. Sei
 $g\in\RR^\Omega$ mit $g\perp_\pi\mathbf{1}$, $\norm{g}_\pi\neq0$. Dann hat
 $\widetilde{g}\coloneqq g/\norm{g}_\pi$ Norm 1 und 
 \begin{equation*}
  \diri(\widetilde{g})=\sprod{(I-P)\fracnorm{g}{g}{\pi},\fracnorm{g}{g}{\pi}}_\pi=\frac{\diri(g)}{\norm{g}_\pi^2}\, .
 \end{equation*}
 Also gilt auch \enquote{$\leq$}.
\end{proof}

\section{Abschätzung der Spektrallücke über die Leitfähigkeit}
Wir formulieren jetzt das Hauptresultat:
\begin{thmbk}[Jerrum und Sinclair (1989), Lawler und Sokal (1988)]
 \label{thm:haupt}
 Sei $P$ eine irreduzible, reversible Übergangsmatrix mit stationärer Verteilung $\pi$.
 Sei $\lambda_2$ der zweitgrößte Eigenwert von $P$ und $\phi_\star$ die Leitfähigkeit.
 Dann gilt für die Spektrallücke $\gamma=1-\lambda_2$:
 \begin{equation}
  \frac{\phi_\star^2}{2}\leq\gamma\leq 2\phi_\star\, .
 \end{equation}
\end{thmbk}

\begin{rem}
 Die Abschätzung nach oben ist linear in $\phi_\star$, die Abschätzung nach
 unten ist quadratisch. Die folgenden Beispiele zeigen, dass es Fälle gibt in
 denen die obere Abschätzung die richtige Ordnung hat, und welche, in denen die untere
 Abschätzung die richtige Ordnung hat. 
\end{rem}
\begin{exmp}[Träge Irrfahrt auf dem n-dimensionalen Hyperwürfel]
 Sei $\Omega=\{-1,1\}^n$ der n-dimensionale Hyperwürfel.
 Betrachte die Menge $S\coloneqq\Set{x\in\Omega\mid x_1=-1}$ der Ecken, deren
 erste Koordinate $-1$ ist. Die stationäre Verteilung ist die Gleichverteilung
 $\pi=(2^{-n},2^{-n},\dotsc,2^{-n})^\transpose$. Also gilt 
 \begin{equation*}
  \pi(S)=2^{-n}\abs{S}=2^{-n}\cdot 2^{n-1}=\frac{1}{2}\, .
 \end{equation*}
 Es sei $E$ die Menge der Kanten.
 Es ist $\{x,y\}\in E$ genau dann, wenn sich $x$ und $y$ in genau einer Koordinate
 unterscheiden. Der Rand von S ist deshalb
 \begin{align*}
  \partial S&=\Set{\{x,y\}\in E\mid x\in S \text{ und } y\in\stcomp{S}} \\
            &=\Set{\{(-1,v), (1,v)\}\mid v\in\{-1,1\}^{n-1}}\, .
 \end{align*}
 Folglich ist $\abs{\partial S}=2^{n-1}$. Die Leitfähigkeit der Menge S lässt
 sich nun berechnen:
 \begin{align*}
  \phi(S)&=\frac{Q(S,\stcomp{S})}{\pi(S)}=2\sum_{x\in S,y\in\stcomp{S}}\pi(x)P(x,y) \\
  &=2\cdot\abs{\partial S}\cdot 2^{-n}\cdot \frac{1}{2n}=2\cdot2^{n-1}\cdot 2^{-n}\cdot \frac{1}{2n}
  =\frac{1}{2n}
 \end{align*}
 Es folgt 
 \begin{equation}
  \label{eq:phistar}
  2\phi_\star\leq 2\phi(S)=\frac{1}{n} \, .
 \end{equation} 
Für jede Teilmenge $J\subset\{1,\dotsc,n\}$ definiere $f_\smallJ:\Omega\to\RR$ durch
 \begin{equation*}
  f_\smallJ(x_1,\dotsc,x_n)\coloneqq\prod_{j\in J}x_j \, .
 \end{equation*}
 \begin{claim}
  $f_\smallJ$ ist Eigenfunktion von $P$ zum Eigenwert $1-\abs{J}/n$.
 \end{claim}
 \begin{dproof}
  Sei $x\in\Omega$ beliebig. Partitioniere $\Omega$ durch vier Mengen:
  \begin{align*}
  A&\coloneqq\Set{y\in\Omega\mid\text{$y$ unterscheidet sich von $x$ nur in einer Koordinate aus $J$}} \\ 
  B&\coloneqq\Set{y\in\Omega\mid\text{$y$ unterscheidet sich von $x$ nur in einer Koordinate aus $\stcomp{J}$}} \\
  C&\coloneqq\{x\} \\
  D&\coloneqq\Set{y\in\Omega\mid\text{$y$ unterscheidet sich von $x$ in mehr als einer Koordinate}}
  \end{align*}
  Für $y\in D$ ist $P(x,y)=0$. Außerdem ist $\abs{A}=\abs{J}$ und $\abs{B}=\abs{\stcomp{J}}=n-\abs{J}$.
  Ändert man eine Koordinate von $x$, die in $J$ enthalten ist, so dreht sich
  das Vorzeichen von $f_\smallJ$ um: $f_\smallJ(y)=-f_\smallJ(x)$. Ändert man aber eine Koordinate
  aus $\stcomp{J}$, so hat dies keinen Einfluss auf $f_\smallJ$, d.h. $f_\smallJ(y)=f_\smallJ(x)$.
  Wir können also rechnen
  \begin{align*}
  Pf(x)&=\sum_{y\in\Omega}P(x,y)f(y) \\
  &=P(x,x)f(x)+\sum_{y\in A}P(x,y)f(y)+\sum_{y\in B}P(x,y)f(y) \\
  &=\frac{1}{2}f(x)+\abs{J}\cdot\frac{1}{2n}\cdot(-f(x))+(n-\abs{J})\frac{1}{2n}f(x) \\
  &=(1-\frac{\abs{J}}{n})f(x)\, .
  \end{align*}
 \end{dproof}
 \begin{claim}
  Es gibt keine weiteren Eigenwerte.
 \end{claim}
 \begin{dproof}
  Zeige zunächst, dass für $I,J\subset\Omega$ mit $I\neq J$ gilt: $\sprod{f_\smallI,f_\smallJ}_\pi=0$.
  Sei dazu o.B.d.A. $k\in I\setminus J$. Dann haben wir
  \begin{align*}
   \sprod{f_\smallI,f_\smallJ}_\pi&=\sum_{x\in\Omega}f_\smallI(x)f_\smallJ(x)\pi(x) \\
   &=2^{-n}\sum_{(x_1,\dotsc,\cancel{x_k},\dotsc,x_n)}\sum_{x_k\in\{-1,1\}}x_k\cdot f_{I\setminus\{k\}}(x)f_\smallJ(x) \\
   &=0\, ,
  \end{align*}
  wobei die letzte Gleichheit gilt, da weder $f_{I\setminus\{k\}}$ noch $f_\smallJ$ von
  $x_k$ abhängen. 
  Die Menge
  \begin{equation*}
   Z\coloneqq\Set{f_\smallJ\mid J\subset\{1,\dotsc,n\}} 
  \end{equation*}
  ist demnach linear unabhängig und $\abs{Z}=2^n=\dim(\RR^\Omega)$, also ist $Z$
  eine Basis von $\RR^\Omega$. 

  Wäre nun $\mu$ ein anderer Eigenwert von $P$, so
  % wäre nach \cref{rem:eigortho} die zugehörige Eigenfunktion $g$
  % $\pi$\=/orthogonal auf allen Eigenvektoren zu anderen Eigenwerten, also auch auf
  % allen Elementen aus $Z$. Dann 
  wäre die zugehörige Eigenfunktion linear unabhängig zu $Z$. Dies ist ein Widerspruch.

  % Da $Z$ eine Basis ist, existiert für jedes $J\subset\{1,\dotsc,n\}$ ein
  % $a_J\in\RR$, sodass
  % \begin{equation*}
  %  g=\sum_{J\subset\{1,\dotsc,n\}}a_Jf_J\, . 
  % \end{equation*}
  % Da $g\neq 0$ ist, existiert $K\subset\{1,\dotsc,n\}$ mit $a_K\neq 0$.
  % Dann gilt
  % \begin{equation*}
  %  \sprod{Pg,f_K}_\pi=\sum_{J\subset\{1,\dotsc,n\}}a_J(1-\frac{\abs{J}}{n})\sprod{f_J,f_K}_\pi
  %  =a_K(1-\frac{\abs{K}}{n}) \,
  % \end{equation*}
  % Andererseits ist
  % \begin{equation*}
  %  \sprod{Pg,f_K}_\pi=\sprod{\mu g,f_K}_\pi=\mu a_K 
  % \end{equation*}
 \end{dproof}
 Wie wir gesehen haben, ist der zweitgrößte Eigenwert der Übergangsmatrix
 $\lambda_2=\nobreak 1-\nobreak 1/n$ und die Spektrallücke ist $\gamma=1-\lambda_2=1/n$. Wegen
 \begin{equation*}
  \frac{1}{n}=\gamma\overset{\ref{thm:haupt}}{\leq} 2\phi_\star\overset{\eqref{eq:phistar}}{\leq}\frac{1}{n} 
 \end{equation*}
 gilt 
 \begin{equation*}
  \gamma=\frac{1}{n}=2\phi_\star 
 \end{equation*}
 und die obere Abschätzung aus \cref{thm:haupt} ist scharf und von korrekter
 Ordnung in $n$.
\end{exmp}
\begin{exmpbk}[Träge Irrfahrt auf dem $2n$\=/Zykel]
 Setze $\omega\coloneqq\exp(i\pi/n)$.
 Sei $\Omega=\Set{\omega^j\mid 0\leq j<2n}$ die Menge der $2n$ Punkte mit
 regelmäßigem Abstand auf dem komplexen Einheitskreis. $\Omega$ bildet mit der
 komplexen Multiplikation eine Gruppe und wir können eine Übergangsmatrix durch
 Angabe der Inkrementenverteilung $\mu$ definieren: 
 \begin{align*}
  \mu(x)\coloneqq
  \begin{cases}
   1/2&\text{für }x=1, \\
   1/4&\text{für }x\in\{\omega,\omega^{-1}\}, \\
   0&\text{sonst.}
  \end{cases}
 \end{align*}
 Die Matrix ist symmetrisch, also ist sie reversibel bezüglich der
 Gleichverteilung auf $\Omega$.
 Für jede Menge $S\subset\Omega$ ist die Leitfähigkeit
 \begin{equation*}
  \phi(S)=\frac{Q(S,\stcomp{S})}{\pi(S)}=\frac{\abs{\partial S}\left( \frac{1}{4} \right)\left( \frac{1}{2n} \right)}{\frac{\abs{S}}{2n}} \, .
 \end{equation*}
 Der Nenner wird unter allen $S$ mit $\pi(S)\leq 1/2$ maximal für $\abs{S}=n$.
 % Ist $\pi(S)\leq 1/2$, so ist $\abs{S}\leq n$.
 Für jede echte Teilmenge $S\subsetneq\Omega$ gilt $\abs{\partial S}\geq 2$.
 Die Menge $S=\{\omega,\omega^2,\dotsc,\omega^n\}$ hat folglich minimale
 Leitfähigkeit und wir haben
 \begin{equation}
  \label{eq:cyclestar}
  \phi_\star=\frac{2\left( \frac{1}{4} \right)\left( \frac{1}{2n} \right)}{\frac{n}{2n}}=\frac{1}{2n} 
 \end{equation}
 Für $0\leq j<2n$ definiere $f_j:\Omega\to\CC, f_j(x)\coloneqq x^j$.
 Dann gilt für alle $k\in\{0,\dotsc,2n-1\}$
 \begin{align*}
  Pf_j(\omega^k)&=\frac{1}{2}f_j(\omega^k)+\frac{1}{4}\left( f_j(\omega^{k-1})+f_j(\omega^{k+1}) \right) \\
  &=\omega^{jk}\left(\frac{1}{2} + \frac{\omega^j+\omega^{-j}}{4} \right)=\frac{1+\cos(\pi\frac{j}{n})}{2}f_j(\omega^k) \, .
 \end{align*}
 Diese Eigenfunktionen sind zwar komplexwertig, aber da die Matrix und die
 Eigenwerte reell sind, gibt es auch reelle Eigenfunktionen (z.\,B. der Realteil $\Re(f_j)$).
 Außerdem sind
 für $i\neq j, i\neq 2n-j$ die Eigenwerte zu $f_i,f_j$ verschieden und die
 Eigenfunktionen $f_j,f_{2n-j}$ ($j\notin\{0,n\}$) sind linear unabhängig.
 Deshalb ist die Menge $\{f_j\mid 0\leq j<2n\}$ eine Basis von $\CC^\Omega$ und es
 gibt keine weiteren Eigenwerte.
 Der zweitgrößte ist 
 \begin{equation*}
  \lambda_2=\frac{1+\cos(\pi/n)}{2}\, =1-\frac{\pi^2}{4n^2}+\mathcal{O}(n^{-4})\qquad\text{für $n\to\infty$.}
 \end{equation*}
 Die Spektrallücke ergibt sich zu $\gamma=\pi^2/(4n^2)+\mathcal{O}(n^{-4})$.
 \Cref{thm:haupt} liefert die untere Schranke
 \begin{equation*}
  \gamma\geq\frac{\phi_\star^2}{2}\overset{\eqref{eq:cyclestar}}{=}\frac{1}{8n^2} 
 \end{equation*}
 mit der korrekten Ordnung in $n$.
\end{exmpbk}

Wir wollen nun \cref{thm:haupt} beweisen.
\begin{proof}[Beweis der oberen Schranke in \cref{thm:haupt}]
 Für jedes $S\subset\Omega$ mit $0<\pi(S)\leq 1/2$ definiere eine Funktion $g_\smallS$ durch 
 \begin{equation*}
  g_\smallS(x)\coloneqq
  \begin{cases}
   -\pi(\stcomp{S})&\text{für }x\in S, \\
   \pi(S)&\text{für }x\notin S.
  \end{cases}
 \end{equation*}
 Dann ist
 \begin{align*}
  E_\pi(g_\smallS)&=\sum_{x\in\Omega}g_\smallS(x)\pi(x)=-\pi(\stcomp{S})\sum_{x\in S}\pi(x)+\pi(S)\sum_{x\in\stcomp{S}}\pi(x) \\
  &=-\pi(\stcomp{S})\pi(S)+\pi(S)\pi(\stcomp{S})=0\, .
 \end{align*}
 Außerdem
 \begin{align*}
  \norm{g_\smallS}_\pi^2&=\sum_{x\in\Omega}g_\smallS(x)^2\pi(x)=\pi(\stcomp{S})^2\pi(S)+\pi(S)^2\pi(\stcomp{S}) \\
  &=\pi(\stcomp{S})\pi(S)(\pi(\stcomp{S})+\pi(S))=\pi(\stcomp{S})\pi(S)
 \end{align*}
 und nach \cref{lem:diri} gilt
 \begin{align*}
  \diri(g_\smallS)=\frac{1}{2}\sum_{x,y\in\Omega}(g_\smallS(x)-g_\smallS(y))^2\pi(x)P(x,y) \, .
 \end{align*}
 Da $g_\smallS$ auf $S$ und $\stcomp{S}$ konstant ist, ist für $x,y\in S$ bzw.
 $x,y\in\stcomp{S}$ $(g_\smallS(x)-g_\smallS(y))^2=0$. Wegen Reversibilität sind die Fälle
 $x\in S,y\in\stcomp{S}$ und $y\in S,x\in\stcomp{S}$ symmetrisch. Übrig bleibt
 \begin{align*}
  \diri(g_\smallS)&=\frac{1}{2}\cdot 2\sum_{x\in S,y\in\stcomp{S}}(g_\smallS(x)-g_\smallS(y))^2\pi(x)P(x,y) \\
  &=\sum_{x\in S,y\in\stcomp{S}}(\pi(S)+\pi(\stcomp{S}))^2\pi(x)P(x,y) \\
  &=Q(S,\stcomp{S})
 \end{align*}
 Laut \cref{lem:spekdiri} ist
 \begin{align*}
  \gamma&=\min\Set{\frac{\diri(g)}{\norm{g}_\pi^2}\mid g\in\RR^\Omega,\, g\perp_\pi\mathbf{1},\, g\neq 0} \\
  &\leq\frac{\diri(g_\smallS)}{\norm{g_\smallS}_\pi^2}=\frac{Q(S,\stcomp{S})}{\pi(S)\pi(\stcomp{S})} \, .
 \end{align*}
 Aus $\pi(S)\leq 1/2$ folgt $\pi(\stcomp{S})\geq 1/2$, deshalb
 \begin{align*}
  \gamma\leq\frac{2Q(S,\stcomp{S})}{\pi(S)}=2\phi(S) \, .
 \end{align*}
 Da dies für alle $S\subset\Omega$ mit $0<\pi(S)\leq 1/2$ gilt, ist auch
 \begin{align*}
  \gamma\leq 2\phi_\star\, . 
 \end{align*}
\end{proof}
Vor dem Beweis der unteren Schranke brauchen wir das folgende Lemma:
\begin{lem}
 \label{lem:ordn}
 Sei $\psi$ eine nichtnegative Funktion auf $\Omega$. Ordne $\Omega$ so, dass
 $\psi$ monoton fällt. Wenn $\pi\{\psi>0\}\leq 1/2$, dann gilt
 \begin{equation*}
  E_\pi(\psi)\leq\phi_\star^{-1}\sum_{\substack{x,y\in\Omega\\x<y}}(\psi(x)-\psi(y))Q(x,y)\, .
 \end{equation*}
\end{lem}
\begin{proof}
 Seit $t>0$ beliebig. Definiere $S\coloneqq\Set{x\in\Omega\mid \psi(x)>t}$. \\
 Es ist ${\pi(S)\leq\pi\{\psi>0\}\leq 1/2}$. Falls $\pi(S)>0$, so gilt
 \begin{equation*}
  \phi_\star\leq\frac{Q(S,\stcomp{S})}{\pi(S)}=\frac{\sum_{x,y\in\Omega}Q(x,y)\mathbbm{1}_{\{\psi(x)>t\geq\psi(y)\}}}{\pi\{\psi>t\}} 
 \end{equation*}
 Nach Wahl der Ordnung ist $\psi(x)>\psi(y)$ nur wenn $x<y$, deshalb
 \begin{equation}
  \label{eq:psi}
  \pi\{\psi>t\}\leq\phi_\star^{-1}\sum_{x<y}Q(x,y)\mathbbm{1}_{\{\psi(x)>t\geq\psi(y)\}}\, .
 \end{equation}
 Da die rechte Seite nichtnegativ ist, gilt diese Ungleichung auch für
 $\pi(S)=0$. Es gilt
 \begin{equation*}
  E_\pi(\psi)=\int_0^\infty\pi\{\psi>t\}\,\mathrm{d}t
 \end{equation*}
 und
 \begin{equation*}
  \psi(x)-\psi(y)=\int_0^\infty\mathbbm{1}_{\{\psi(x)>t\geq\psi(y)\}}\,\mathrm{d}t\, ,
 \end{equation*}
 also gibt integrieren von \cref{eq:psi} über t
 \begin{equation*}
  E_\pi(\psi)\leq\phi_\star^{-1}\sum_{x<y}(\psi(x)-\psi(y))Q(x,y)\, .
 \end{equation*}
\end{proof}
\begin{proof}[Beweis der unteren Schranke in \cref{thm:haupt}]
 Sei $f_2$ eine Eigenfunktion zum Eigenwert $\lambda_2$. Ohne Einschränkung sei
 $\pi\{f_2>0\}\leq 1/2$ (sonst wechsle auf $-f_2$). \\
 Definiere $f\coloneqq\max\{f_2,0\}$.
 Nach \cref{rem:eigortho} ist $f_2\perp_\pi\mathbf{1}$, was nur möglich ist, wenn
 $f_2$ sowohl positive als auch negative Einträge hat. $f$ ist deswegen nicht die Nullfunktion.
 \begin{claim}
  Für alle $x\in\Omega$ gilt $((I-P)f)(x)\leq\gamma f(x)$.
 \end{claim}
 \begin{dproof}
  Sei $x\in\Omega$ beliebig. Falls $f(x)=0$, so gilt aufgrund der Nichtnegativität von $f$
  \begin{equation*}
   ((I-P)f)(x)=0-\sum_{y\in\Omega}P(x,y)f(y)\leq 0=\gamma f(x)\, . 
  \end{equation*}
  Falls $f(x)>0$, so gilt wegen $f\geq f_2$
  \begin{align*}
   ((I-P)f)(x)&=f_2(x)-\sum_{y\in\Omega}P(x,y)f(x) \\
   &\leq f_2(x)-\sum_{y\in\Omega}P(x,y)f_2(x)=(1-\lambda_2)f_2(x)=\gamma f(x)
  \end{align*} 
 \end{dproof}
 Mit \cref{lem:diri} und weil $f$ nichtnegativ ist, folgt
 \begin{align*}
  \begin{split}
   \diri(f)=\sprod{(I-P)f,f}_\pi&=\sum_{x\in\Omega}((I-P)f)(x)f(x)\pi(x) \\
   &\leq \sum_{x\in\Omega}\gamma f(x)^2\pi(x)=\gamma\sprod{f,f}_\pi\, .
  \end{split}
 \end{align*}
 Wie oben bemerkt ist $f$ nicht konstant Null, wir können also umstellen zu
 \begin{equation}
  \label{eq:R}
  \gamma\geq\frac{\diri(f)}{\sprod{f,f}_\pi}\eqqcolon R\, . 
 \end{equation}
 Wir wollen \cref{lem:ordn} für $\psi=f^2$ anwenden. Dies ist möglich, denn
 $\pi\{f^2>0\}=\pi\{f_2^2>0\}\leq 1/2$. Nach entsprechender Wahl der Ordnung auf
 $\Omega$ haben wir
 \begin{equation*}
  \sprod{f,f}_\pi^2=E_\pi(f^2)^2\leq\phi_\star^{-2}\left[ \sum_{x<y}(f(x)^2-f(y)^2)Q(x,y) \right]^2
 \end{equation*}
 Mit der Cauchy--Schwarz-Ungleichung bekommt man
 \begin{equation}
  \label{eq:cauchy}
  \sprod{f,f}_\pi^2\leq\phi_\star^{-2}\left[ \sum_{x<y}(f(x)-f(y))^2Q(x,y) \right]\left[ \sum_{x<y}(f(x)+f(y))^2Q(x,y) \right]\, .
 \end{equation}
 Es gilt
 \begin{equation*}
  (f(x)+f(y))^2=2f(x)^2+2f(y)^2-(f(x)-f(y))^2 
 \end{equation*}
 und der rechte Faktor in \cref{eq:cauchy} lautet
 \begin{align}
  \sum_{x<y}(f(x)-f(y))^2Q(x,y)=2&\sum_{x<y}(f(x)^2+f(y)^2)Q(x,y) \label{eq:eins} \\
  +&\sum_{x<y}(f(x)-f(y))^2Q(x,y)\, . \label{eq:zwei}
 \end{align}
 Wegen Reversibilität ist $Q(x,y)=Q(y,x)$ und es gilt für den ersten Term
 \eqref{eq:eins} die Abschätzung
 \begin{align*}
  \sum_{x<y}(f(x)^2+f(y)^2)Q(x,y)&=\sum_{x\in\Omega}f(x)^2\pi(x)\sum_{\substack{y\in\Omega\\x<y}}P(x,y)+\sum_{y\in\Omega}f(y)^2\pi(y)\sum_{\substack{x\in\Omega\\x<y}}P(y,x) \\
  &=\sum_{x\in\Omega}f(x)^2\pi(x)(1-P(x,x))\leq\sprod{f,f}_\pi \, .
 \end{align*}
 Im ersten Faktor von \cref{eq:cauchy} wie auch im zweiten Term aus
 \cref{eq:zwei} fallen Summanden mit $x=y$ weg, deswegen finden wir die
 Dirichletform wieder:
 \begin{align*}
  \sum_{x<y}(f(x)-f(y))^2Q(x,y)&=\frac{1}{2}\left( \sum_{x<y}(f(x)-f(y))^2Q(x,y)+\sum_{y<x}(f(x)-f(y))^2Q(x,y)\right)\\
  &=\frac{1}{2}\sum_{x.y\in\Omega}(f(x)-f(y))^2Q(x,y) \\
  &=\diri(f)\, .
 \end{align*}
 Insgesamt ist \cref{eq:cauchy} äquivalent zu
 \begin{equation*}
  \sprod{f,f}_\pi^2\leq\phi_\star^{-2}\cdot\diri(f)\cdot \left(2\sprod{f,f}_\pi-\diri(f)\right) \, .
 \end{equation*}
 Teile durch $\sprod{f,f}_\pi^2$ und stelle um zu
 \begin{equation*}
  \phi_\star^2\leq R(2-R)\, ,
 \end{equation*}
 d.h.
 \begin{equation*}
  1-\phi_\star^2\geq1-2R+R^2=(1-R)^2\geq(1-\gamma)^2\, , 
 \end{equation*}
 wobei im letzten Schritt auf \cref{eq:R} zurückgegriffen wurde. Schließlich gilt
 \begin{equation*}
  \left( 1-\frac{\phi_\star^2}{2} \right)^2=1-\phi_\star^2+\frac{\phi_\star^4}{4}\geq 1-\phi_\star^2\geq (1-\gamma)^2\, , 
 \end{equation*}
 und die untere Schranke $\phi_\star^2/2\leq\gamma$ ist bewiesen.
\end{proof}
\printbibliography

\end{document}